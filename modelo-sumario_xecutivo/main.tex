\documentclass{acm_proc_article-sp}

\usepackage{graphicx,url}
\usepackage[brazil,english]{babel}   
\usepackage[utf8]{inputenc}
\usepackage{paralist}

\newcommand{\fig}[4][htb]{
  \begin{figure}[#1]
    {\centering{\includegraphics[#4]{fig/#2}}\par}
    \caption{#3}
    \label{fig:#2}
  \end{figure}
}

\newcommand{\figtwo}[8]{
\begin{figure}[h]
	\centering
	\begin{subfigure}{#8}
		\centering
		\includegraphics[#3]{fig/#1}
	\end{subfigure}%
	\begin{subfigure}{#8}
		\centering
		\includegraphics[#6]{fig/#4}
	\end{subfigure}
\end{figure}
}

\newcommand{\tab}[4][h]{
  \begin{table}[h]
    {\centering\footnotesize\textsf{\input{fig/#2.tab}}\par}
    \caption{#3}
    \label{tab:#2}
  \end{table}
}

\begin{document}

\title{Sistema de Monitoramento e Alertas Emergenciais para Idosos}
\subtitle{\vspace{-2ex} -- Proposta de Projeto --\\
Instituto Federal de Santa Catarina campus São José\\
Projeto Integrador II - Engenharia de Telecomunicações}

\numberofauthors{2}

\author{
\alignauthor
Andrey Gonçalves, Bruno Martins do Nascimento, Daniel Valdeley Marques, Victor Cesconeto de Pieri\\
\email{andrey.g@aluno.ifsc.edu.br}
}

\maketitle

\begin{abstract}
  Com o envelhecimento da população surgiram lacunas no cuidado e monitoramento de idosos. 
  Com o avanço tecnológico viabilizou a criação de sistemas para monitoramento. Tendo em vista isso, a proposta do projeto é apresentar uma solução para auxiliar o monitoramento de pessoas idosas e seu ambiente domiciliar. A ideia principal consiste no desenvolvimento de um gateway MQTT utilizando uma Raspberry Pi 4 com um módulo de comunicação \textit{Zigbee} para processar os sinais de diversos sensores \textit{Zigbee} de mercado e uma \textit{stack} de um cliente SIP para comunicação de voz em casos emergenciais ou para assistência ao contratante.
  As informações de monitoramento do idoso e do ambiente será exibida em um \textit{Dashboard} na plataforma TagoIO com uma interface amigável, onde o encarregado pelo monitoramento pode verificar em tempo real os dados coletados no ambiente de instalação. Vale ressaltar que os dados coletados para desenvolvimento do sistema são de sensores de temperatura, humidade, fumaça, gás de cozinha, porta, botão de emergência, vibração para detecção de quedas, vazamento de água e presença. Além disso o produto tem duas botoeiras onde pode ser acionada para ligar para números pre-definidos, como uma central de atendimento de saúde ou algum familiar. O produto possui alto falante e microfone embutido e é capaz de originar e receber chamadas SIP.
\end{abstract}


\section{Desafio}

Desta forma, a solução oferece um conjunto completo de serviço de soluções de monitoramento remoto para ajudar os idosos a manter um vida saudável e segura. Podendo ter serviços de tele assistência projetados especificamente para ajudar os idosos a manter sua independência e viver com segurança na sua casa.

\section{Solução Proposta}

A ideia principal consiste
no desenvolvimento de um gateway Message Queuing Telemetry Transport (MQTT)
utilizando uma Raspberry Pi 4 com um módulo de comunicação Zigbee para processar os
sinais de diversos sensores Zigbee de mercado e uma stack de um cliente Session Initiation
Protocol (SIP) para comunicação de voz em casos emergenciais ou para assistência ao
contratante. As informações de monitoramento do idoso e do ambiente será exibida em um
Dashboard na plataforma TagoIO com uma interface amigável, onde o encarregado pelo
monitoramento pode verificar em tempo real os dados coletados no ambiente de instalação.

\fig{cat}{A skeptical cat...}{width=0.8\columnwidth}

\section{Principais Tecnologias}

Dê uma visão geral sobre a(s) principal(is) tecnologia(s) envolvida(s) na solução do projeto.
Quais são os principais conhecimentos que você precisará empregar nesta solução (redes, sistemas operacionais, comunicação, etc)?
Brevemente, descreva as tecnologias e conhecimentos apontando seu papel no projeto.

Bla bla bla bla bla bla bla bla bla bla bla bla bla bla bla bla bla bla.
Bla bla bla bla bla bla bla bla bla bla bla bla bla bla bla bla bla bla.
Bla bla bla bla bla bla bla bla bla bla bla bla bla bla bla bla bla bla.
Bla bla bla bla bla bla bla bla bla bla bla bla bla bla bla bla bla bla.
Bla bla bla bla bla bla bla bla bla bla bla bla bla bla bla bla bla bla.
Bla bla bla bla bla bla bla bla bla bla bla bla bla bla bla bla bla bla.
Bla bla bla bla bla bla bla bla bla bla bla bla bla bla bla bla bla bla.
Bla bla bla bla bla bla bla bla bla bla bla bla bla bla bla bla bla bla.
Bla bla bla bla bla bla bla bla bla bla bla bla bla bla bla bla bla bla.

\section{Riscos}

O que pode dar errado no seu projeto?
Qual é a probabilidade de que estes problemas realmente ocorram?
O que pode ser feito para reduzir estes riscos?

Bla bla bla bla bla bla bla bla bla bla bla bla bla bla bla bla bla bla.
Bla bla bla bla bla bla bla bla bla bla bla bla bla bla bla bla bla bla.
Bla bla bla bla bla bla bla bla bla bla bla bla bla bla bla bla bla bla.
Bla bla bla bla bla bla bla bla bla bla bla bla bla bla bla bla bla bla.
Bla bla bla bla bla bla bla bla bla bla bla bla bla bla bla bla bla bla.
Bla bla bla bla bla bla bla bla bla bla bla bla bla bla bla bla bla bla.
Bla bla bla bla bla bla bla bla bla bla bla bla bla bla bla bla bla bla.
Bla bla bla bla bla bla bla bla bla bla bla bla bla bla bla bla bla bla.
Bla bla bla bla bla bla bla bla bla bla bla bla bla bla bla bla bla bla.

\section{Orçamento}

Quais equipamentos/materiais vocês precisarão para executar seu projeto?
Quanto isto vai custar?

\begin{table}[h!]
\centering
\begin{tabular}{l|c|r}
Item & Qtd & Preço (R\$) \\
\hline
Arduino & 5 & 250.00 \\
Drone   & 1 & 2000.00 \\
Fios    & 1 & 20.00 \\
\hline
\multicolumn{2}{r}{Total} & 2270.00
\end{tabular}
\end{table}


\section{Cronograma}

Quanto tempo o projeto vai levar?
Quais são as principais datas e o que será entregue em cada data?

\begin{table}[h!]
\centering
\begin{tabular}{l|c|r}
Atividade & Inicio & Fim (R\$) \\
\hline
Projeto & 14/fev & 04/jul \\
\hline
Concepção do projeto & 14/fev & 05/mar \\
Detalhamento do projeto & 07/mar & 04/abr \\
Aquisição e Testes de equipamentos & 07/mar & 04/abr \\
Atividade X & ?? & ?? \\
Atividade Y & ?? & ?? \\
Atividade Z & ?? & ?? \\
\end{tabular}
\label{tab:my_label}
\end{table}


\end{document}
