\documentclass{acm_proc_article-sp}

\usepackage{graphicx,url}
\usepackage[brazil,english]{babel}   
\usepackage[utf8]{inputenc}
\usepackage{paralist}

\newcommand{\fig}[4][htb]{
  \begin{figure}[#1]
    {\centering{\includegraphics[#4]{fig/#2}}\par}
    \caption{#3}
    \label{fig:#2}
  \end{figure}
}

\newcommand{\figtwo}[8]{
\begin{figure}[h]
	\centering
	\begin{subfigure}{#8}
		\centering
		\includegraphics[#3]{fig/#1}
	\end{subfigure}%
	\begin{subfigure}{#8}
		\centering
		\includegraphics[#6]{fig/#4}
	\end{subfigure}
\end{figure}
}

\newcommand{\tab}[4][h]{
  \begin{table}[h]
    {\centering\footnotesize\textsf{\input{fig/#2.tab}}\par}
    \caption{#3}
    \label{tab:#2}
  \end{table}
}

\begin{document}

\title{Título do Projeto}
\subtitle{\vspace{-2ex} -- Proposta de Projeto --\\
Instituto Federal de Santa Catarina campus São José\\
Projeto Integrador II - Engenharia de Telecomunicações}

\numberofauthors{2}

\author{
\alignauthor
Autor01, Autor02, Autor03, Autor04\\
\email{email do lider do projeto (apenas um membro do grupo gerenciando comunicações)}
}

\maketitle

\begin{abstract}
Resumo descrevendo os objetivos principais e o impacto do projeto.
\end{abstract}


\section{Desafio}

Descreva o principal problema que o projeto tentará resolver.
Em geral, tente responder à pergunta: o que este projeto está resolvendo?
Não esqueça de explicar porque solucionar este problem é importante!

Bla bla bla bla bla bla bla bla bla bla bla bla bla bla bla bla bla bla.
Bla bla bla bla bla bla bla bla bla bla bla bla bla bla bla bla bla bla.
Bla bla bla bla bla bla bla bla bla bla bla bla bla bla bla bla bla bla.
Bla bla bla bla bla bla bla bla bla bla bla bla bla bla bla bla bla bla.
Bla bla bla bla bla bla bla bla bla bla bla bla bla bla bla bla bla bla.
Bla bla bla bla bla bla bla bla bla bla bla bla bla bla bla bla bla bla.
Bla bla bla bla bla bla bla bla bla bla bla bla bla bla bla bla bla bla.
Bla bla bla bla bla bla bla bla bla bla bla bla bla bla bla bla bla bla.
Bla bla bla bla bla bla bla bla bla bla bla bla bla bla bla bla bla bla.

\section{Solução Proposta}

Esta sessão deve descrever como você planeja resolver o problema apresentado na sessão anterior.
É uma ótima idéia usar alguma figura ou diagrama para apresentar a solução proposta.

\fig{cat}{A skeptical cat...}{width=0.8\columnwidth}

Bla bla bla bla bla bla bla bla bla bla bla bla bla bla bla bla bla bla.
Bla bla bla bla bla bla bla bla bla bla bla bla bla bla bla bla bla bla.
Bla bla bla bla bla bla bla bla bla bla bla bla bla bla bla bla bla bla.
Bla bla bla bla bla bla bla bla bla bla bla bla bla bla bla bla bla bla.
Bla bla bla bla bla bla bla bla bla bla bla bla bla bla bla bla bla bla.
Bla bla bla bla bla bla bla bla bla bla bla bla bla bla bla bla bla bla.
Bla bla bla bla bla bla bla bla bla bla bla bla bla bla bla bla bla bla.
Bla bla bla bla bla bla bla bla bla bla bla bla bla bla bla bla bla bla.
Bla bla bla bla bla bla bla bla bla bla bla bla bla bla bla bla bla bla.

\section{Principais Tecnologias}

Dê uma visão geral sobre a(s) principal(is) tecnologia(s) envolvida(s) na solução do projeto.
Quais são os principais conhecimentos que você precisará empregar nesta solução (redes, sistemas operacionais, comunicação, etc)?
Brevemente, descreva as tecnologias e conhecimentos apontando seu papel no projeto.

Bla bla bla bla bla bla bla bla bla bla bla bla bla bla bla bla bla bla.
Bla bla bla bla bla bla bla bla bla bla bla bla bla bla bla bla bla bla.
Bla bla bla bla bla bla bla bla bla bla bla bla bla bla bla bla bla bla.
Bla bla bla bla bla bla bla bla bla bla bla bla bla bla bla bla bla bla.
Bla bla bla bla bla bla bla bla bla bla bla bla bla bla bla bla bla bla.
Bla bla bla bla bla bla bla bla bla bla bla bla bla bla bla bla bla bla.
Bla bla bla bla bla bla bla bla bla bla bla bla bla bla bla bla bla bla.
Bla bla bla bla bla bla bla bla bla bla bla bla bla bla bla bla bla bla.
Bla bla bla bla bla bla bla bla bla bla bla bla bla bla bla bla bla bla.

\section{Riscos}

O que pode dar errado no seu projeto?
Qual é a probabilidade de que estes problemas realmente ocorram?
O que pode ser feito para reduzir estes riscos?

Bla bla bla bla bla bla bla bla bla bla bla bla bla bla bla bla bla bla.
Bla bla bla bla bla bla bla bla bla bla bla bla bla bla bla bla bla bla.
Bla bla bla bla bla bla bla bla bla bla bla bla bla bla bla bla bla bla.
Bla bla bla bla bla bla bla bla bla bla bla bla bla bla bla bla bla bla.
Bla bla bla bla bla bla bla bla bla bla bla bla bla bla bla bla bla bla.
Bla bla bla bla bla bla bla bla bla bla bla bla bla bla bla bla bla bla.
Bla bla bla bla bla bla bla bla bla bla bla bla bla bla bla bla bla bla.
Bla bla bla bla bla bla bla bla bla bla bla bla bla bla bla bla bla bla.
Bla bla bla bla bla bla bla bla bla bla bla bla bla bla bla bla bla bla.

\section{Orçamento}

Quais equipamentos/materiais vocês precisarão para executar seu projeto?
Quanto isto vai custar?

\begin{table}[h!]
\centering
\begin{tabular}{l|c|r}
Item & Qtd & Preço (R\$) \\
\hline
Arduino & 5 & 250.00 \\
Drone   & 1 & 2000.00 \\
Fios    & 1 & 20.00 \\
\hline
\multicolumn{2}{r}{Total} & 2270.00
\end{tabular}
\label{tab:my_label}
\end{table}


\section{Cronograma}

Quanto tempo o projeto vai levar?
Quais são as principais datas e o que será entregue em cada data?

\begin{table}[h!]
\centering
\begin{tabular}{l|c|r}
Atividade & Inicio & Fim (R\$) \\
\hline
Projeto & 14/fev & 04/jul \\
\hline
Concepção do projeto & 14/fev & 05/mar \\
Detalhamento do projeto & 07/mar & 04/abr \\
Aquisição e Testes de equipamentos & 07/mar & 04/abr \\
Atividade X & ?? & ?? \\
Atividade Y & ?? & ?? \\
Atividade Z & ?? & ?? \\
\end{tabular}
\label{tab:my_label}
\end{table}


\end{document}
