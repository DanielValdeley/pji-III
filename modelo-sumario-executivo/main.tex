\documentclass{acm_proc_article-sp}

\usepackage{graphicx,url}
\usepackage[brazil,english]{babel}   
\usepackage[utf8]{inputenc}
\usepackage{paralist}

\newcommand{\fig}[4][htb]{
  \begin{figure}[#1]
    {\centering{\includegraphics[#4]{fig/#2}}\par}
    \caption{#3}
    \label{fig:#2}
  \end{figure}
}

\newcommand{\figtwo}[8]{
\begin{figure}[h]
	\centering
	\begin{subfigure}{#8}
		\centering
		\includegraphics[#3]{fig/#1}
	\end{subfigure}%
	\begin{subfigure}{#8}
		\centering
		\includegraphics[#6]{fig/#4}
	\end{subfigure}
\end{figure}
}

\newcommand{\tab}[4][h]{
  \begin{table}[h]
    {\centering\footnotesize\textsf{\input{fig/#2.tab}}\par}
    \caption{#3}
    \label{tab:#2}
  \end{table}
}

\begin{document}

\title{Sistema de Monitoramento e Alertas Emergenciais para Idosos}
\subtitle{\vspace{-2ex} -- Proposta de Projeto --\\
Instituto Federal de Santa Catarina campus São José\\
Projeto Integrador II - Engenharia de Telecomunicações}

\numberofauthors{2}

\author{
\alignauthor
Andrey Gonçalves, Bruno Martins do Nascimento, Daniel Valdeley Marques, Victor Cesconeto de Pieri\\
\email{andrey.g@aluno.ifsc.edu.br}
}

\maketitle

\begin{abstract}
  Com o envelhecimento da população surgiram lacunas no cuidado e monitoramento de idosos. 
  Com o avanço tecnológico viabilizou a criação de sistemas para monitoramento. Tendo em vista isso, a proposta do projeto é apresentar uma solução para auxiliar o monitoramento de pessoas idosas e seu ambiente domiciliar. A ideia principal consiste no desenvolvimento de um gateway MQTT utilizando uma Raspberry Pi 4 com um módulo de comunicação \textit{Zigbee} para processar os sinais de diversos sensores \textit{Zigbee} de mercado e uma \textit{stack} de um cliente SIP para comunicação de voz em casos emergenciais ou para assistência ao contratante.
  As informações de monitoramento do idoso e do ambiente será exibida em um \textit{Dashboard} na plataforma TagoIO com uma interface amigável, onde o encarregado pelo monitoramento pode verificar em tempo real os dados coletados no ambiente de instalação. Vale ressaltar que os dados coletados para desenvolvimento do sistema são de sensores de temperatura, humidade, fumaça, gás de cozinha, porta, botão de emergência, vibração para detecção de quedas, vazamento de água e presença. Além disso o produto tem duas botoeiras onde pode ser acionada para ligar para números pre-definidos, como uma central de atendimento de saúde ou algum familiar. O produto possui alto falante e microfone embutido e é capaz de originar e receber chamadas SIP.
\end{abstract}


\section{Desafio}

Desta forma, a solução oferece um conjunto completo de serviço de soluções de monitoramento remoto para ajudar os idosos a manter um vida saudável e segura. Podendo ter serviços de tele assistência projetados especificamente para ajudar os idosos a manter sua independência e viver com segurança na sua casa.

\section{Solução Proposta}

A ideia principal consiste
no desenvolvimento de um gateway Message Queuing Telemetry Transport (MQTT)
utilizando uma Raspberry Pi 4 com um módulo de comunicação Zigbee para processar os
sinais de diversos sensores Zigbee de mercado e uma stack de um cliente Session Initiation
Protocol (SIP) para comunicação de voz em casos emergenciais ou para assistência ao
contratante. As informações de monitoramento do idoso e do ambiente será exibida em um
Dashboard na plataforma TagoIO com uma interface amigável, onde o encarregado pelo
monitoramento pode verificar em tempo real os dados coletados no ambiente de instalação.

\fig{zigbee}{Aplicações para o ZigBee.}{width=0.8\columnwidth}

\section{Principais Tecnologias}

A solução proposta nesse artigo é baseado em tecnologia digital, com integração
completa pela internet, o que traz ao produto maior confiabilidade e velocidade para
a tomada de decisões.O produto possuirá um intercomunicador
IP que permite enviar ligações para uma central de atendimento ou algum contato pré
determinado. Através dos sensores, é monitorado possíveis situações de risco no ambiente
ou com o usuário do sistema e estes dados são enviados via rádio comunicação, Zigbee, que possui curto alcance e baixo consumo, para o intercomunicador que imediatamente reporta como um alarme para a central. Um dashboard de
monitoramento para verificar os dados coletados pelos sensores será desenvolvido utilizando
a plataforma de nuvem TagoIO.
Além disso o produto será um comunicador que utiliza o protocoloca da camada de aplicação SIP, que é caracterizado pela sua fácil implementação, com alto falante e
microfone embutido, onde é possível originar uma chamada de emergência ou até mesmo
requisitar algum tipo de atendimento com uma central de monitoramento de saúde. Isso será feito através do desenvolvimento de um gateway MQTT, que possui foco em Internet das Coisas em rede TCP/IP.


\section{Riscos}


Risco de custo dos componentes. Devido à variação de preços e utilização de hardwares específicos, há chance de ocorrer mudança no orçamento. A solução é fazer uma estimativa realista para manter o projeto.



\section{Orçamento}

\begin{table}[!h]
\ABNTEXfontereduzida
\centering
\caption{Custos para o protótipo}
\begin{tabular}{cc}
\hline
\textbf{Produto}                & \textbf{Preço (Reais)} \\ \hline
Raspberry Pi 4 model b          & 729,90                 \\
Alto falante USB P2             & 30,30                  \\
Mini Microfone USB 2.0          & 40                     \\
Case plástico                   & 50                     \\
Sonoff Zigbee 3.0 USB           & 144                    \\
Sensor de gás de cozinha        & 152                    \\
Sensor de fumaça                & 149                    \\
Sensor de vibração              & 101                    \\
Sensor de temperatura e umidade & 101,18                 \\
Botão de emergência             & 86                     \\
Sensor de porta                 & 151                    \\
Sensor de alagamento            & 106                    \\
Sensor de movimento             & 136                    \\
Honorários x 4p 4m              & 1.800                   \\ \hline
\textbf{TOTAL}                  & R\$ 30.776,38           \\ \hline
\end{tabular}
\fonteproprioautor

\end{table}


\section{Cronograma}


\begin{table}[h!]
\centering
\begin{tabular}{l|c|r}
\\Atividade & Inicio & Fim \\
\hline
Projeto & 08/fev & 22/jun \\
\hline
Estudo de Viabilidade & 08/fev & 15/fev \\
Entrega hardware, & & \\
Software de configuração da Rasp & 16/fev & 15/abr \\
Envio do Zigbee para Cloud & 16/abr & 25/mai \\
Documentação técnica & 26/mai & 22/jun \\
\end{tabular}
\label{tab:my_label}
\end{table}


\end{document}
